\section{Plano de Trabalho e Cronograma}\label{sec:cronograma}

Este trabalho tem como objetivo desenvolver e implementar algoritmos numéricos para a solução de equações diferenciais ordinárias (EDOs) utilizando a linguagem de programação \emph{Python} e as bibliotecas numéricas \emph{NumPy} e \emph{Numba}. Para alcançar esse objetivo, foi realizada uma revisão bibliográfica a fim de adquirir uma fundação teórica no domínio do problema, com foco no estudo de equações diferenciais e métodos iterativos explícitos e implícitos.

Para verificar a precisão e confiabilidade do código, os testes do algoritmo começarão com problemas mais simples, com soluções analíticas ou numéricas conhecidas. Em seguida, serão realizados testes preliminares para verificar o código, usando diferentes condições iniciais e comparando os resultados com a literatura existente. Finalmente, será realizada uma análise dos erros obtidos, da convergência e do tempo computacional.

Durante todo o processo, será redigido um relatório das atividades para documentar e apresentar os resultados obtidos. O uso do compilador JIT \emph{Numba} também será considerado para melhorar o desempenho computacional do algoritmo.

Este projeto de pesquisa foi desenvolvido em um período de doze meses e dividido em oito etapas:

\begin{enumerate}
	\item Revisão bibliográfica: realização de pesquisa bibliográfica em livros e artigos científicos para adquirir fundamentação teórica no domínio do problema.
	\item Implementação das equações diferenciais em linguagem computacional: codificação dos modelos matemáticos das equações diferenciais ordinárias, utilizando as bibliotecas \emph{Numpy}, \emph{Numba} e a linguagem de programação \emph{Python}.
	\item Estudo e implementação dos métodos numéricos iterativos e de diferenças finitas: estudo e implementação de métodos numéricos iterativos e de diferenças finitas para solucionar as equações diferenciais implementadas.
	\item Estudo e uso de um método de solução de sistema de equações não lineares: estudo e implementação de um método de solução de sistema de equações não lineares para tratar equações diferenciais implícitas.
	\item Testes preliminares para constatação da acurácia do código: realização de testes com problemas mais simples para verificar a precisão do código implementado.
	\item Testes numéricos para validação dos métodos iterativos: execução de testes numéricos para validar os métodos implementados, utilizando diferentes condições iniciais e sendo comparados com resultados da literatura.
	\item Análise dos resultados obtidos: análise dos erros obtidos, da convergência e do tempo computacional dos métodos numéricos iterativos e de diferenças finitas implementados.
	\item Relatório Final: redação do relatório final das atividades realizadas, contendo a descrição detalhada dos métodos implementados e resultados obtidos.
	\end{enumerate}
\section{Trabalhos Relacionados} \label{related}

A Tabela \ref{related-works} apresenta, de forma resumida, os principais trabalhos relacionados ao tema deste trabalho, apresentando os seus objetivos, métodos e resultados.

\begin{table}[h]
\centering
\caption{Trabalhos relacionados ao tema desta revisão.}
\vspace{0.5cm}
\label{related-works}
\begin{tabularx}{\linewidth}{|p{4cm}|l|X|}
\hline
\textbf{Referência} & \textbf{Ano} & \textbf{Descrição} \\
\hline
Image Inpainting \cite{Bertalmio2000} & 2000 & Baseia-se no uso de amostras sem a necessidade de serem explicitamente selecionados pelo usuário. O algoritmo os seleciona automaticamente e completa as regiões utilizando síntese de textura. O algoritmo possui limitações ao ser utilizado em grandes regiões com texturas complexas, pois não consegue reproduzir a textura sem causar borramentos. \\

\hline

An Image Inpainting Technique Based on the Fast Marching Method \cite{Telea2004} & 2004 & Este trabalho visa obter resultados satisfatórios com um algortimo mais simples e rápido. O autor propõe preencher a região a ser recuperada ponderando a região vizinha, assim como apresentado por \cite{Bertalmio2000}. Em suma, o método poposto produz resultados satisfatórios para remoção de textos e arranhões e, segundo o autor, possui suas limitações quando a região a ser preenchida se torna maior que 10-15 pixels. \\

\hline



\hline

\end{tabularx}
\end{table}








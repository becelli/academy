\section{Referências bibliográficas}
\begin{thebibliography}{99}
	
\bibitem{Bertalmio2000}
Bertalmio, M., Sapiro, G., Caselles, V., e Ballester, C. (2000). "Image inpainting". Proceedings of the 27th Annual Conference on Computer Graphics and Interactive Techniques — SIGGRAPH ’00.

\bibitem{Elharrouss2019}
Elharrouss, O., Almaadeed, N., Al-Maadeed, S., Akbari, Y. (2019). Image Inpainting: A Review. Neural Processing Letters.

\bibitem{Fontoura2022}
Fontoura, C. (2022). “Estudos De Métricas Para Quantificar Os Resultados De Aplicação De inpainting Para Melhoria Da Qualidade Do Processo De Extração De Feições Em Imagens Digitais”. Relatório de Qualificação de Doutorado, FCT-UNESP.

\bibitem{Dolhansky2018}
B. Dolhansky and C. C. Ferrer, "Eye In-painting with Exemplar Generative Adversarial Networks," 2018 IEEE/CVF Conference on Computer Vision and Pattern Recognition, 2018, pp. 7902-7911, doi: 10.1109/CVPR.2018.00824.

\bibitem{Bertalmio2001}
Bertalmio, M., Sapiro, G., Caselles, V.,  Ballester, C. (2001). Image inpainting. In Computer Graphics and Applications, 2001. Proceedings. 21st International Conference on (pp. 417-424). IEEE.

\bibitem{Criminisi2004}
Criminisi, A., Pérez, P., Toyama, K. (2004). Region filling and object removal by exemplar-based image inpainting. IEEE Transactions on image processing, 13(9), 1200-1212.

\bibitem{Efros1999}
Efros, A. A., Leung, T. K. (1999). Texture synthesis by non-parametric sampling. In Proceedings of the 26th annual conference on Computer graphicvs and interactive techniques (pp. 1033-1038). ACM.


\end{thebibliography}

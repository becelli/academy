% Apresentar uma panorâmica da área a ser estudada, descrevendo o histórico da área, o cenário atual, os problemas existentes, o objetivo deste trabalho e, no final, a estrutura deste documento, descrevendo brevemente os demais capítulos desta revisão.

\section{Introdução} \label{introduction}
O Inpainting  de Imagens é o processo de preencher regiões faltantes ou danificadas de uma imagem com o intuito de restaurá-la ou melhorar sua aparência. Este é um problema amplamente estudado em visão computacional e possui uma vasta gama de oportunidades de aplicações, incluindo o restauro de obras artísticas e fotografias danificadas, a remoção de objetos de imagens e síntese de texturas~\cite{criminisi2004region}.

Esta área de pesquisa tem sido estudada há várias décadas. Os primeiros trabalhos sobre o tema surgiram na década de 1950, mas foi somente a partir dos anos 1990 que o inpainting começou a ser utilizado de maneira mais ampla.

\cite{Elharrouss2019} apresenta uma revisão sobre o inpainting de imagens, descrevendo os principais métodos e aplicações. Neste campo, existem várias técnicas para o inpainting de imagens, cada tal com suas próprias vantagens e desvantagens. Uma abordagem popular é apresentada em \cite{Bertalmio2000}, a qual introduz uma técnica de inpainting baseada no conceito de complementar uma região de uma imagem existente. Neste trabalho é utilizado equações diferenciais parciais para propagar informações das bordas da região danificada para o centro da região, permitindo que o algoritmo estime como os pixels devem ser restaurados.

Uma das contribuições significativas para o campo de inpainting de imagem é o desenvolvimento do trabalho de ~\cite{Telea2004}, um algoritmo eficiente que usa a informação das fronteiras para propagar informações de pixels conhecidos para os pixels desconhecidos na região de inpainting. Este algoritmo tem sido amplamente utilizado em várias aplicações, devido a ser implementado em algumsa bibliotecas de software, incluindo a biblioteca de código aberto de visão computacional OpenCV \cite{OpenCV}. Este método tem demonstrado produzir resultados de alta qualidade para uma ampla variedade de tipos de imagem e cenários, e se tornou uma das referências para avaliar o desempenho de outros algoritmos de inpainting.

Nos últimos anos, houve um aumento de interesse em utilizar métodos baseados em aprendizado de máquina, como redes adversárias gerativas (GANs) e redes neurais convolucionais (CNNs) \cite{pathakCVPR16context}, para resolver o problema de inpainting. Esses métodos têm o potencial de aprender padrões e estruturas complexas a partir de grandes conjuntos de dados, o que pode ser usado para gerar resultados de inpainting de alta qualidade. No entanto, esses métodos podem ser computacionalmente caros e podem exigir uma abundância de dados de treinamento para obter bons resultados.

Hoje, o inpainting é utilizado em uma variedade de aplicações. Dentre elas, está incluída o restauro de obras de arte danificadas, a remoção de objetos de imagens (como nuvens em imagens de sensoriamento remoto) e até mesmo a
remoção de imperfeições de fotografias. Além disso, pode-se utilizar os mesmos conceitos de restauração para modificar as perspectivas de imagens, como é apresentado em \cite{huang2014image}.

O principal problema encontrado neste campo de pesquisa são a escolha do método de inpainting mais adequado para cada tipo de imagem e cenário. Este fator se deve às desvantagens que os métodos existentes proporcionam \cite{Salem2021}. Em geral, os maiores desafios são:
\begin{itemize} 
  \item \textbf{Preservação da consistência e estrutura da imagem de entrada:} muitos modelos geradores podem gerar imagens visualmente agradáveis, mas que não correspondem perfeitamente com a vizinhança ou à textura da imagem original.
  \item \textbf{Preservação de detalhes finos:} este problema é particularmente desafiador para imagens com texturas complexas, como imagens de paisagens naturais, ou a área que está sendo restaurada é muito grande.
  \item \textbf{Reconstruir objetos sobrepostos:} quando um objeto é parcial ou completamente sobreposto, como um objeto que está coberto por uma nuvem, o modelo gerador pode não ser capaz de reconstruir o objeto original.
  \item \textbf{Preenchimento de grandes regiões:} a geração de imagens de alta qualidade quando há uma grande região a ser reconstruída ainda é um problema desafiador.
  \item \textbf{Desempenho computacional:} a maioria dos métodos de inpainting de imagens que geram imagens de alta qualidade são computacionalmente caros, dificultando sua aplicação em tempo real.
\end{itemize}


\subsection{Objetivos}

\subsubsection{Objetivo principal}
O objetivo principal deste Trabalho de Conclusão de Curso é investigar e implementar algumas das diferentes técnicas de Inpainting para a restauração de imagens digitais com defeitos ou objetos indesejados. Para alcançar esse objetivo, serão estudadas técnicas clássicas baseadas em informações de vizinhança e amostras, bem como técnicas mais recentes baseadas em aprendizado de máquina. Além disso, serão aplicadas algumas das técnicas estudadas em uma variedade de imagens e comparar os resultados obtidos.

\subsubsection{Objetivos secundários}
\begin{itemize}
  \item Estudar alguns dos principais métodos de inpainting, com o intuito de entender como cada um deles funciona e quais são suas vantagens e desvantagens.
  \item Aplicar as técnicas de inpainting em algum escopo, ou, para algoritmos generalistas, em diversos tipos de imagens, incluindo imagens coloridas, imagens de alta resolução e imagens com diferentes tipos de defeitos, como buracos, riscos, manchas, etc.
  \item Aperfeiçoar os conhecimentos em processamento de imagens e aprendizado de máquina do aluno, bem como a capacidade de aplicar esses conhecimentos em problemas reais.
  \item Estudar a possibilidade de melhorias na eficácia (qualidade das imagens geradas) e na eficiência (tempo de processamento, uso de memória) dos métodos de inpainting estudados.
\end{itemize}

Os três objetivos secundários representam o âmbito deste trabalho dentro do objetivo principal. O último, por sua vez, representa o interesse do aluno em aprofundar seus conhecimentos em otimização computacional de algoritmos e a possibilidade de contribuir com a comunidade científica.

\subsection{Organização do Trabalho}
Inicialmente apresenta-se uma fundamentação teórica sobre o processo de inpainting, incluindo uma breve introdução sobre o escopo de inpainting, as principais categorias da área e posteriormente uma revisão de alguns dos principais métodos utilizados.

Em seguida, há uma seção destinada a mencionar algumas possíveis técnicas de otimização computacionais que podem ser aplicadas caso haja a possibilidade de serem implementadas.

Por fim, é apresentado um levantamento de trabalhos e artigos científicos que se relacionam com o tema do trabalho, bem como uma breve descrição de cada um deles.
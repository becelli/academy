\section*{Resumo}
O inpainting de imagens é a técnica de modificar imagens com aplicação majoritariamente concentrada em restauração de imagens corrompidas por ruídos, borrões, arranhões, objetos indesejados, textos sobrepostos (e.g. legendas e marcas-d’-água), defeitos da lente de captura, entre diversas outras falhas em imagens digitais. 

Os métodos estabelecidas para aplicar o inpainting podem ser divididos em duas categorias: (1) métodos sequenciais clássicos, e (2) métodos baseados em aprendizado, que utilizam redes neurais convolucionais (CNNs) e/ou Redes Generativas Adversárias (GANs). A primeira categoria ordinariamente possui métodos mais leves, que recorrem às equações diferenciais parciais (PDEs), enquanto as duas últimas, para obterem melhores resultados, utilizam maior processamento computacional e necessitam de treinamento prévio para ajustar seus parâmetros.

Este trabalho desenvolverá este tema e apresentará uma revisão teórica dos métodos de inpainting, aprentado algumas das técnicas e conceitos principais utilizados. Como resultado, foi possível compreender que os métodos clássicos são, em geral, mais rápidos e eficientes, mas apresentam resultados menos realistas e borramentos. Por outro lado, os métodos baseados em aprendizado são mais lentos e computacionalmente mais custosos, mas apresentam resultados mais plausíveis. Apesar disto, em geral, todos costumam considerar as regiões vizinhas ou semelhantes para preencher a região alvo.

Palavras-chave: Inpainting, processamento de imagens, preenchimento de regiões, aprendizado de máquina.
\pagebreak
\section*{Abstract}
Image inpainting is a technique for modifying images with application mainly focused on restoring images corrupted by noise, blurs, scratches, unwanted objects, superimposed texts (e.g. subtitles and watermarks), capture lens defects, among others. several other flaws in digital images.

The established methods for applying inpainting can be divided into two categories: (1) classical sequential methods, and (2) learning-based methods, which use convolutional neural networks (CNNs) and/or Generative Adversarial Networks (GANs). The first category ordinarily has lighter methods, which resort to partial differential equations (PDEs), while the last two, in order to obtain better results, use more computational processing and require prior training to adjust their parameters.

This work will develop this theme and present a theoretical review of inpainting methods, presenting some of the techniques and main concepts used. As a result, it was possible to understand that the classical methods are, in general, faster and more efficient, but present less realistic results and blurring. On the other hand, learning-based methods are slower and computationally more expensive, but present more plausible results. Despite this, in general, everyone usually considers neighboring or similar regions to fill the target region.

Keywords: Inpainting, image processing, region filling, machine learning.
% Apresentar uma panorâmica da área a ser estudada, descrevendo o histórico da área, o cenário atual, os problemas existentes, o objetivo deste trabalho e, no final, a estrutura deste documento, descrevendo brevemente os demais capítulos desta revisão.

Inpainting é o processo de preencher partes faltantes ou danificadas de uma imagem ou vídeo. O inpainting de imagem é uma área de pesquisa que tem sido estudada há várias décadas. Os primeiros trabalhos sobre o tema surgiram na década de 1950, mas foi somente a partir dos anos 1990 que o inpainting começou a ser utilizado de maneira mais ampla.

Uma das primeiras técnicas de inpainting foi desenvolvida por @! James Crimmins em 1992. Sua técnica utilizava uma combinação de síntese de textura e interpolação para preencher as partes faltantes de uma imagem. A técnica de Crimmins foi aprimorada por outros pesquisadores @!, incluindo o uso de técnicas de aprendizado de máquina para estimar a textura de uma região danificada.

Em 1997, Bertalmio, Marcelo, Andrea L. Bertozzi e Guillermo Sapiro @! introduziram uma técnica de inpainting baseada no conceito de complementar a imagem existente. Sua técnica utilizava equações parciais diferenciais para propagar informações das bordas da região danificada para o centro da região, permitindo que o algoritmo estimasse como os pixels faltantes deveriam parecer.

Na década de 2000, várias outras técnicas de inpainting foram propostas, incluindo técnicas baseadas em wavelets @!, codificação esparsa @! e aprendizado profundo @!. Essas técnicas têm continuado a melhorar ao longo do tempo e se tornaram cada vez mais eficazes na síntese de imagens e vídeos com aparência realista.

Hoje, o inpainting é utilizado em uma variedade de aplicações, incluindo o restauro de obras de arte danificadas, a remoção de objetos de imagens e vídeos e até mesmo a remoção de imperfeições de fotografias. É uma ferramenta importante no campo de processamento de imagem e vídeo e continua sendo uma área de pesquisa ativa.
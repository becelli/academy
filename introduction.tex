% Apresentar uma panorâmica da área a ser estudada, descrevendo o histórico da área, o cenário atual, os problemas existentes, o objetivo deste trabalho e, no final, a estrutura deste documento, descrevendo brevemente os demais capítulos desta revisão.

O Inpainting de Imagens é o processo de preencher regiões faltantes ou danificadas de uma imagem ou vídeo para restaurar ou melhorar sua aparência. Este é um problema amplamente estudado em visão computacional e possui uma vasta gama de aplicações, incluindo o restauro de obras de arte e fotografias danificadas, a remoção de objetos de imagens e síntese de texturas (Criminisi, Pérez, Toyama, 2004).

Esta área de pesquisa tem sido estudada há várias décadas. Os primeiros trabalhos sobre o tema surgiram na década de 1950, mas foi somente a partir dos anos 1990 que o inpainting começou a ser utilizado de maneira mais ampla.

Nos últimos anos, houve um aumento de interesse em utilizar métodos baseadas em aprendizado, como redes adversárias gerativas (GANs) e redes neurais convolucionais (CNNs), para resolver o problema de inpainting. Esses métodos têm o potencial de aprender padrões e estruturas complexas a partir de grandes conjuntos de dados, o que pode ser usado para gerar resultados de inpainting de alta qualidade (Pathak et al., 2016; Yu et al., 2018; Nazeri et al., 2019). No entanto, esses métodos podem ser computacionalmente caros e podem exigir grandes quantidades de dados de treinamento para obter bons resultados (Liu et al., 2020).

Existem várias abordagens para o inpainting de images, cada uma com suas próprias vantagens e desvantagens. Uma abordagem popular é o inpainting baseado em amostras, o qual usa informações de pixels vizinhos para gerar novos pixels para preencher as regiões faltantes ou danificadas (Hertzmann et al., 2001). Essa abordagem funciona bem para imagens com texturas simples, mas pode falhar quando a textura da imagem é complexa ou quando a região danificada é muito grande (Criminisi et al., 2004).

Uma das contribuições mais significativas para o campo de inpainting de imagem é o desenvolvimento do "Método de Marcha Rápida" (Telea, 2004), um algoritmo eficiente que usa equações diferenciais parciais para propagar informações de pixels conhecidos para os pixels desconhecidos na região de inpainting. Este algoritmo tem sido amplamente utilizado em várias aplicações e foi implementado em muitas bibliotecas de software, incluindo a biblioteca aberta de visão computacional OpenCV (OpenCV, n.d.). Este método tem demonstrado produzir resultados de alta qualidade para uma ampla variedade de tipos de imagem e cenários, e se tornou uma das referências para avaliar o desempenho de outros algoritmos de inpainting.

Hoje, o inpainting é utilizado em uma variedade de aplicações, incluindo o restauro de obras de arte danificadas, a remoção de objetos de imagens e vídeos e até mesmo a remoção de imperfeições de fotografias. É uma ferramenta importante no campo de processamento de imagem e vídeo e continua sendo uma área de pesquisa ativa.
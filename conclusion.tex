% Conclua o que foi apresentado. Apresente as perspectivas futuras para o desenvolvimento 
% do trabalho, a partir do exposto nesta revisão.

\section{Conclusão} \label{conclusion}
Neste trabalho, foram revisados vários conceitos e trabalhos sobre o inpainting de imagens, incluindo os métodos clássicos e os baseados em aprendizado de máquina. Recapitulando, os métodos clássicos geralmente realizam o uso da difusão das regiões de fronteiras e amostras de outras regiões da imagem. Possuem a vantagem de serem computacionalmente baratas e podem produzir bons resultados para imagens com texturas e estruturas simples. Por outro lado, os métodos baseados em aprendizado utilizam técnicas de aprendizado de máquina para preencher a região alvo e podem produzir resultados mais realistas, incluindo aquelas com texturas e estruturas mais complexas. Contudo, estes métodos geralmente são mais complexos de serem implementados e treinados, e computacionalmente mais custosos.

Baseado nesta revisão, será dado início a implementação de alguns dos métodos existentes, com o objetivo de aplicar a fundamentação teórica obtida relacionada ao inpainting de imagens. Após esta etapa, será realizada uma comparação dos resultados obtidos quando aplicados a uma variedade de imagens com diferentes texturas, estruturas e contextos. Por fim, será realizada uma análise dos resultados obtidos, com o objetivo de identificar os pontos fortes e fracos de cada método, e caso oportuno, propor melhorias para os mesmos.